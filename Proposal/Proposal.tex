\documentclass[11pt]{article}

\usepackage{amsmath}
\usepackage{amssymb}
\usepackage[margin = 2.5cm]{geometry}
\usepackage{graphicx}

\usepackage{sectsty}
\sectionfont{\fontsize{12}{16}\selectfont\centering}
\subsectionfont{\fontsize{10}{14}\selectfont}

\usepackage{natbib}
\bibliographystyle{apalike}

\usepackage{hyperref}
\hypersetup{colorlinks,linkcolor={blue},citecolor={blue},urlcolor={blue}}  


\author{Gregor Steiner}
\title{Master Thesis Proposal}

\begin{document}

\maketitle

\section*{Idea}

The aim of this paper is to estimate the causal effect of extreme heat events on fatal car accidents. It is well known that excessive heat increases aggression and irritability \citep{anderson2011implications}, both of which are qualities that are likely to decrease driving performance. Thus, it is plausible that heat increases the likelihood of accidents.

The economic impact is also substantial. In 2017, the cost of medical care and the productivity losses associated from injuries and deaths from motor vehicle accidents exceeded \$75 billion\footnote{These numbers are available on the \href{https://www.cdc.gov/transportationsafety/costs/index.html}{CDC website}.}. Therefore, even small changes in the number of accidents can have substantial economic impacts, which makes this area of research highly relevant.

\section*{Data}

I combine weather data provided by the Daily Global Historical Climatology Network \citep{Menne_2012} and data on fatal car accidents by \citet{Smith_2016}. The weather data is collected by more than 1200 measurement stations across the US and they can be aggregated to the county level using the geographic information provided in the stations' metadata. Thus, I have a panel dataset containing daily county-level data on fatal car accidents and weather from 1990 to 2013.

\section*{Empirical Strategy}

Following \citet{Habeeb_2015}, I define an extreme heat event as two or more
consecutive days in which the maximum or minimum temperature exceeds the 85th percentile of July and August maximum or minimum temperatures. 

My next step is thinking about the identification of the causal effect. By surveying similar studies, I will try to identify potential confounders. While variation in weather is quasi-random, the high availability of weather forecasts may induce substantial anticipation effects. For example, people may shift their leisure activities to and therefore drive more on warmer days, which might lead to more fatal accidents, even without a causal effect of the heat itself. Thus, there may be selection effects despite quasi-random treatment assignment.

A Poisson panel model may be an option, as the number of rare events within a given time period (like fatal car accidents per day) tends to follow a Poisson distribution. Another approach I find very interesting is utilizing matrix completion methods to estimate the ATT, e.g. as proposed by \citet{Athey_2021}.






\bibliography{references}

\end{document}
