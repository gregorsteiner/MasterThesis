\documentclass[11pt]{article}

\usepackage{amsmath}
\usepackage{amssymb}
\usepackage[margin = 2.5cm]{geometry}
\usepackage{graphicx}

\usepackage{sectsty}
\sectionfont{\fontsize{12}{16}\selectfont\centering}
\subsectionfont{\fontsize{10}{14}\selectfont}

\usepackage{natbib}
\bibliographystyle{apalike}

\usepackage{hyperref}
\hypersetup{colorlinks,linkcolor={blue},citecolor={blue},urlcolor={blue}}  


\author{Gregor Steiner}
\title{Master Thesis Proposal}

\begin{document}

\maketitle

\section*{Idea \& Data}

My goal is to estimate the causal effect of heatwaves or extreme heat events on fatal car accidents. I combine weather data provided by the Daily Global Historical Climatology Network \citep{Menne_2012} and data on fatal car accidents by \citet{Smith_2016}. The weather data is measured by more than 1200 stations across the US and they can be aggregated to the county level using the geographic information provided in the stations' metadata. Thus, I have a panel dataset containing at least 10 years of daily county-level data on car accidents and weather.

\section*{Empirical Strategy}

My next step is thinking about the identification of the causal effect. By surveying the literature on physiological effects of heat, I will try to identify potential confounders. Variation in weather is quasi-random, but due to the high availability of weather forecasts, there may be substantial anticipation effects. For example, people may shift their leisure activities to and therefore drive more on warmer days, which might lead to more fatal accidents, even without a causal effect.

A Poisson panel model may be an option, as the number of rare events (like fatal car accidents) tends to follow a Poisson distribution. Another possible approach is utilizing matrix completion methods to estimate the ATT, e.g. as proposed by \citet{Athey_2021}.






\bibliography{references}

\end{document}
