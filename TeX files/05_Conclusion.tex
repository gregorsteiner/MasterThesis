\section{Conclusion} \label{Conclusion}

The main contribution of this paper is estimating dynamic treatment effects for natural disasters on academic achievement. In most specifications, we find negative short-term effects on the average mathematics test scores. Black and hispanic students, however, do not experience these. Based on the FEMA declarations data, there are positive long-term effects in reading achievement. With the NWS storms data, this effect is only visible among white students.

Some of these results may be driven by migration. Shares of hispanic and black students tend to decrease around the time of disasters, at least based on the FEMA declarations. This might explain why they do not experience such pronounced short-term decreases in test scores. Yet, this is somewhat speculative. For a more detailed analysis of student mobility following a disaster, individual level data would be preferable.

Based on temperature data, we analyze the effect of heat exposure on academic performance. We find some significantly negative effects. Hispanic students seem to be especially affected by excessive heat. These discrepancies are likely driven by differing access to air-conditiong. Providing widespread access to air-conditioning can be an effective policy option to prevent such negative effects.

Both the frequency and the severity of natural disasters will increase in the future. Similarly, heat exposure will substantially increase over the next decades. Minimizing the negative effects will be essential in order not to jeopardize children's success at school. Policymakers must focus on mitigating climate change through effective action. However, some effects will only be able to be mitigated to a very limited extent. That is where effective relief measures after a disaster are needed.




