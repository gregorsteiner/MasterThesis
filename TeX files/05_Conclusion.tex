\section{Conclusion} \label{Conclusion}

This study estimates dynamic effects of natural disasters on academic performance measured by standardized test results in mathematics and Reading Language Arts (RLA). For both datasets we find a negative effect on the performance in mathematics in the year the disaster occured. The effect reaches up to 0.01 or even 0.015 standard deviations of the national reference cohort. For RLA we find no significant effects on the overall mean score.

Based on FEMA natural disaster data we find that the performance in RLA among black students increases substantially in the years following a natural disaster. The reason could be that black students may disproportionately benefit from having to switch schools after a disaster \citep{Sacerdote_2012}. However, the same model estimated on the NWS storm data does not confirm these findings.

In total, there is strong evidence for a negative effect of disasters on performance in mathematics in the same school year. For RLA, on the other hand, there is no significant effect. Evidence for medium and long term effects is weak.  There are some significant effects among minority students, but they do not seem to be very robust, that is they only appear in one of the datasets used.

Based on temperature data, we also analyze the effect of heat exposure on academic performance. For the overall student population, we do not find a significant effect. However, for hispanic and to a weaker degree also black and economically disadvantaged students we find significantly negative effects.

Mitigating such negative effects should be a concern for policymakers. Even if effect sizes are small, such negative effects can quickly compound in regions that are frequently affected by disasters. 




