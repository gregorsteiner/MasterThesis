
\section{Empirical Strategy}



In order to identify a causal effect, unobservable determinants of a county's mean test score must be unrelated to natural disasters conditional on observable characteristics of that county. The occurrence of natural disasters is plausibly random conditional on location. Furthermore conditioning on the year should account for an increasing trend in natural disasters due to climate change. Thus, independence of mean test scores and natural disasters is plausible conditional on county and year fixed effects.

Consequently, the baseline specification is
\begin{align} \label{baseline}
	y_{i, t, g} = \sum_{\tau = -9, \tau \neq -1}^{9} \beta_\tau D_{i, t-\tau} + \alpha_i + \lambda_t + \zeta_g + X_{i, t} \gamma + \varepsilon_{i, t, g} \;,
\end{align}
where $y_{i, t, g}$ is the outcome of interest for county $i$, year $t$, and grade $g$. County, year, and grade fixed-effects are given by $\alpha_i$, $\lambda_t$, and $\zeta_g$ respectively and $X_{i, t}$ is a row vector of additional control variables. $D_{i, t-\tau}$ is a treatment indicator for county $i$ in year $t-\tau$. That is, $D_{i, t-\tau} = 1$ if the county experienced a disaster $\tau$ years ago at time $t$.

Note that treatment must be absorbing, meaning the sequence $(D_{i, t})_{t=1}^T$ must be a non-decreasing sequence of $0$s and $1$s. In other words, after being treated for the first time a county stays treated. In the present application this means treatment refers to having experienced a disaster rather than experiencing a disaster in that year. This is common practice and does not cause bias due to the conditionally random nature of natural disasters \citep[see][]{Deryugina_2017}. Thus, this captures the long term effects of having experienced a natural disaster.

It is implausible that the treatment effects are constant in our setting. The extent of disasters varies substantially, and also the level of preparation for such disasters likely displays high variance across counties. 


