
\section{Background \& Data} \label{Data}

\subsection{Natural Disasters in the United States}

Natural disasters cause numerous fatalities and billions of dollars of infrastructure damage in the United States each year \citep{Boustan_2020}. With more greenhouse gas in the atmosphere the number and severity of such disasters will increase \citep{IPCC_2021}. Figure \ref{DisasterCount} shows the number of county-level disasters per year. However, note that a single disaster that affects multiple counties counts as multiple county-level disasters. The majority of natural disasters in the US is made up by storms, such as hurricanes or tornadoes, floodings, and fires. Some well-known examples include the 2020 wildfire season in the western United States, Hurricane Harvey in 2017, or Hurricane Katrina in 2005.

\begin{figure}[!h]
	\centering
	\includegraphics[scale=1]{"../Code & Data/DisasterCount.pdf"}
	\caption{Number of county-level natural disasters by year}
	\label{DisasterCount}
\end{figure}

The handling of natural disasters is governed by the \cite{Stafford}. It defines major disasters as follows:
\begin{quote}
	 Major disaster means any natural catastrophe (including any 	hurricane, tornado, storm, high water, winddriven water, tidal wave, tsunami, earthquake, volcanic eruption, landslide, mudslide, snowstorm, or drought), or, regardless of cause, any fire, flood, or explosion, in any part of the United States, which in the determination of the President causes damage of sufficient severity and magnitude to warrant major disaster assistance under this Act to supplement the efforts and available resources of States, local governments, and disaster relief organizations in alleviating the damage, loss, hardship, or suffering caused thereby.
\end{quote}
Based on the Stafford Act major disaster declarations are made solely by the president upon request by the affected state's governor. State officials must prove that the situation is beyond the capabilities of the state governemnt or involved local governments. The Federal Emergency Management Agency (FEMA) evaluates requests for major disasters and makes recommendations to the president.

A major disaster declaration provides a wide range of federal assistance programs for individuals and public infrastructure, including funds for both emergency and permanent work. In particular, affected state and local governments can receive federal disaster assistance as part of the Public Assistance program, while individuals and households can receive aid from the Individual Assistance program. Moreover, State, Tribal, and local governments and certain private nonprofit organizations can receive assistance for preventive actions taken based on the Hazard Mitigation Assistance program.

FEMA provides data on all federally declared natural disasters, beginning in 1953. The data is easily accessible via their API \citep{rfema}. This is the main source for natural disaster data used here. Figure \ref{DisasterMap} shows the number of declared disasters between 2009 and 2018 across the US. Table \ref{tab:DisasterTypes} shows the types of disasters and their proportion in the FEMA data. Storms make up the largest share of disaster events. Fires and floods also form a substantial part.

\begin{figure}[!h]
	\centering
	\includegraphics[scale=1]{"../Code & Data/DisasterMap.pdf"}
	\caption{Number of declared natural disasters in school years 2008-09 through 2017-18}
	\label{DisasterMap}
\end{figure}

\begin{table}[!htbp] \centering \renewcommand*{\arraystretch}{1.1}\caption{Disasters from 2009 to 2018 by type}\label{DisasterTypes}
\begin{tabular}{lrr}
\hline
\hline
Variable & N & Percent \\ 
\hline
Disaster Type & 13226 &  \\ 
... Chemical & 9 & 0.1\% \\ 
... Coastal Storm & 12 & 0.1\% \\ 
... Dam/Levee Break & 3 & 0\% \\ 
... Earthquake & 19 & 0.1\% \\ 
... Fire & 886 & 6.7\% \\ 
... Flood & 2006 & 15.2\% \\ 
... Freezing & 1 & 0\% \\ 
... Hurricane & 3094 & 23.4\% \\ 
... Mud/Landslide & 28 & 0.2\% \\ 
... Other & 7 & 0.1\% \\ 
... Severe Ice Storm & 803 & 6.1\% \\ 
... Severe Storm(s) & 5644 & 42.7\% \\ 
... Snow & 577 & 4.4\% \\ 
... Tornado & 114 & 0.9\% \\ 
... Toxic Substances & 1 & 0\% \\ 
... Tsunami & 9 & 0.1\% \\ 
... Typhoon & 11 & 0.1\% \\ 
... Volcano & 2 & 0\%\\ 
\hline
\hline
\end{tabular}
\end{table}



We repeat the analysis on two different datasets: Data on storms and data on heat exposure. First, this acts as a robustness check. These two datasets are based on obective measurements, while the FEMA data is based on subjective declarations. Second, this allows us to better understand heterogeneity by type of disaster.

The National Weather Service (NWS) provides data on storm events. In particular, this covers hurricanes, tornadoes, and other severe storms. These make up a very large part of all natural disasters experienced in the United States (see table \ref{DisasterTypes}). Combined they account for more than 80\% of all disaster damage in the FEMA Public Assistance Applicants Program Deliveries database.

\begin{figure}[!h]
	\centering
	\includegraphics[scale=1]{"../Code & Data/StormMap.pdf"}
	\caption{Number of storms in school years 2008-09 through 2017-18}
	\label{StormMap}
\end{figure}

We only consider severe storms which are likely to cause substantial damage. Tornadoes can be classified based on estimated peak wind speeds on the Enhanced Fujita (EF) scale \citep[for more details see][]{EF_Scale}. Tornadoes with an EF scale of 0 or 1 (wind speeds of up to 110 mph) are characterized as weak. Therefore, we exclude those and only keep tornadoes with an EF scale of at least 2 (wind speeds of at least 111 mph). Unfortunately, the hurricane data does not include a similar measure, but it does include an estimated amount of property damage. We exclude all hurricanes with an estimated property damage of zero.  Storm exposure by county is shown in figure \ref{StormMap}.

To measure heat exposure, we exploit daily temperature data from the Global Historical Climatology Network \citep{Menne_2012}. Each county is assigned the measurement station with the lowest distance to the county's center. Following \cite{Goodman_2020}, we use two measures of yearly cumulative heat exposure: The average daily maximum temperature and the number of days with a maximum temperature above 30°C in a school year. The threshold of 30°C is somewhat arbitrary, however, changing it slightly does not change the results in a meaningful way. Figures \ref{HeatMapTemp} and \ref{HeatMapDays} show the distribution across counties. As expected, the variation in heat exposure is largely driven by location.

\begin{figure}[!h]
	\centering
	\includegraphics[scale=1]{"../Code & Data/HeatMapTemp.pdf"}
	\caption{Average daily maximum temperature (in °C) in school years 2008-09 through 2017-18}
	\label{HeatMapTemp}
\end{figure}

\begin{figure}[!h]
	\centering
	\includegraphics[scale=1]{"../Code & Data/HeatMapDays.pdf"}
	\caption{Average number of days above 30°C in school years 2008-09 through 2017-18}
	\label{HeatMapDays}
\end{figure}


\subsection{Assistance Applications}

After severe disasters counties frequently receive public assistance. This potentially creates a selection problem: The counties receiving state aid are likely coping better with the consequences of the natural disaster. If the assignment of federal aid is not independent of the disaster's consequences, this creates a problem for our identification. Since assistance provision is at least somewhat dependent on disaster severity, this is very probable.

FEMA provides a dataset on  their Public Assistance Applicants Program Deliveries. It contains information on applicants and their recovery priorities, including the amount of damage caused and amount of federal disaster assistance granted. This is the main program for federal public disaster assistance, averaging USD 4.7 billion in assistance each year. However, note that counties that do not receive public assistance may still benefit from individual assistance or from the Hazard Mitigation Assistance program.

Unfortunately, this data is only available starting in October 2016. Based on the temporal overlap between this dataset and our main dataset, that is schoolyears 2016/2017 and 2017/2018, it is possible to analyze whether counties that are affected by disasters receive federal assistance. This can be done by checking whether a county that experienced a disaster in 2016/2017 or 2017/2018 appears in the Public Assistance Applicants Program Deliveries database. In fact, only about 17.78\% of counties that experienced a disaster in that period did apply for federal assistance. Table \ref{tab:AppsByType} shows the number of disasters and the share of counties that applied for assistance following such a disaster by disaster type. It seems that the number varies dramatically by disaster type. While about 80\% of counties affected by a tornado applied for federal assistance, only about 10\% of counties affected by fires or floods did so.

\begin{table}

\caption{\label{tab:AppsByType}Share of counties that applied for federal assistance following a disaster by disaster type}
\centering
\begin{tabular}[t]{l|r|r}
\hline
  & Number of Cases & Applied for Assistance (in \%)\\
\hline
Coastal Storm & 1 & 0.00\\
\hline
Dam/Levee Break & 3 & 0.00\\
\hline
Fire & 83 & 2.41\\
\hline
Flood & 110 & 0.00\\
\hline
Hurricane & 1115 & 20.36\\
\hline
Mud/Landslide & 1 & 0.00\\
\hline
Severe Ice Storm & 41 & 0.00\\
\hline
Severe Storm(s) & 220 & 15.00\\
\hline
Tornado & 29 & 79.31\\
\hline
Total & 1603 & 17.78\\
\hline
\end{tabular}
\end{table}


It may be interesting to see how these counties differ from the ones that did apply. Figure \ref{AssistCovBoxplot} shows boxplots by county application status. Counties that did apply for federal disaster assistance tend to have lower median income, higher poverty rates, and higher shares of single motherhood. Thus, it seems that counties that had to apply for federal disaster assistance were more socially vulnerable in the first place. This is consistent with the findings in \cite{Gao_2022}.

However, the direction of causality is not clear. Possibly these counties are more frequently affected by natural disasters and are also poorer or more socially vulnerable because of it. Alternatively, counties that are poorer could be more likely to apply for public disaster aid as they have fewer private resources. While there is some correlation between disaster risk and economic conditions \citep[for example][]{Goodman_2020}, the latter explanation seems more likely overall.

\begin{figure}[!h]
	\centering
	\includegraphics[scale=1]{"../Code & Data/AssistanceCovBoxplot.pdf"}
	\caption{Boxplots by application status}
	\label{AssistCovBoxplot}
\end{figure}

It is also interesting whether variation in the federal assistance procedure may be driven by political factors. Visually, democratic votes in the 2016 election (almost coincides with the start of the Public Assistance Applicants Program Deliveries dataset) tend to be lower in counties that applied. That is, counties that applied for federal assistance tend to vote more Republican. Logistic regression results confirm the visual impression (see Appendix \ref{AppendixA}). While this is not necessarily a causal effect, it could be an indication that a Republican president may be more hesitant awarding disaster assistance to Democratic counties.

There are clear differences between counties that receive federal assistance after a disaster and counties that do not. This is a potential problem to the identification of a causal effect. However, counties that are severely affected and receive assistance would most likely experience even worse consequences, did they not receive assistance. Thus, our estimates may only be a lower bound for the negative effect.


\subsection{Standardized Testing Data}

Data on academic achievement is available from the Stanford Education Data Archive \citep{SEDA}. They provide mean test results from standardized tests by county, year, grade and subject among all students and various subgroups (including race, gender, and economically disadvantaged). The most recent version 4.1 covers grades 3 through 8 in mathematics and Reading Language Arts (RLA)\footnote{RLA assesses students' ability to understand what they read and to write clearly.} over the 2008-09 through 2017-18 school years.

Test scores are cohort-standardized, meaning they can be interpreted relatively to an average national reference cohort in the same grade. This makes the dataset very attractive, as test scores are nationally comparable. For instance, a county mean of 0.5 indicates that the average student in the county scored approximately one half of a standard deviation higher than the average national student in the same grade.

In addition to overall mean test scores, the data includes mean test scores for various subgroups, e.g. by ethnicity. In particular, we consider mean test scores for white, black, hispanic, female, and economically disadvantaged students to investigate whether the effects differ by ethnicity, gender, or socioeconomic position. These are only reported if the subgroups' sample sizes are large enough to guarantee anonymization. Thus, the number of observations for some of the groups is substantially smaller.

The outcomes of interest are overall mean test scores by county, and mean test scores for white, black, hispanic, female, and, economically disadvantaged students. Figure \ref{DepVarsBoxplot} shows boxplots for the five outcomes of interest and Table \ref{tab:SumStats} provides summary statistics. Due to the way the scale is constructed, overall test scores are distributed symmetrically around zero, except for a few outliers. The mean scores for black, hispanic, and economically disadvantaged students tend to be lower than the overall means, while white students tend to perform slightly better than the overall average. Female mean scores are slightly above overall mean scores, meaning that female students perform slightly better than male students on average.

\begin{table}

\caption{\label{tab:SumStats}Summary statistics for dependent variables}
\centering
\begin{tabular}[t]{lrrrrrr}
\toprule
  & Overall & White & Black & Hispanic & Female & Econ. disadv.\\
\midrule
Mean & -0.042 & 0.107 & -0.483 & -0.281 & 0.025 & -0.284\\
Std. Dev. & 0.294 & 0.262 & 0.273 & 0.266 & 0.295 & 0.256\\
Min. & -3.196 & -2.936 & -2.745 & -1.699 & -2.862 & -3.007\\
Max. & 1.669 & 1.700 & 1.394 & 1.374 & 1.496 & 1.312\\
\bottomrule
\end{tabular}
\end{table}


\begin{figure}[!h]
	\centering
	\includegraphics[scale=1]{"../Code & Data/DepVarsBoxplot.pdf"}
	\caption{Boxplots of the outcomes of interest}
	\label{DepVarsBoxplot}
\end{figure}


Natural disasters should only have an effect on test scores if they occur before the test. Standardized tests are generally administered in March, April, or May. We will use March 1st as a cut-off point. Thus, any disaster happening within the same school year before the 1st of March will be considered. School years tend to start in late August or early September with some variation across states. We will use September 1st, meaning any disaster happening between September 1st and March 1st will be counted for a given school year. Disasters occuring in the summer or in the spring after the exams should have much less influence on performance. Thus, we do not consider disasters that occur between March 1st and September 1st.

Each disaster is assigned to a school year as described above. Then, disaster and test score data can be merged by school year and county. This yields a panel data set with six grades and two subjects for each county-year combination.

