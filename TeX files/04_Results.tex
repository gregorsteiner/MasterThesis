\section{Results} \label{Results}

Figure \ref{ResultsPlot} shows the dynamic treatment effects and figure \ref{ResultsPlotSub} shows the same plot for the four subgroups of interest. 

\begin{figure}[!h]
	\centering
	\includegraphics[scale=1]{"../Code & Data/ResultsPlot.png"}
	\caption{Dynamic Treatment effects for all students: FEMA disaster data}
	\label{ResultsPlot}
\end{figure}

\begin{figure}[!h]
	\centering
	\includegraphics[scale=1]{"../Code & Data/ResultsPlotSub.png"}
	\caption{Dynamic Treatment effects for various subgroups: FEMA disaster data}
	\label{ResultsPlotSub}
\end{figure}

For the period of treatment there is a significant\footnote{Significant is used here in the sense that a confidence interval with nominal coverage of 95\% does not include zero, that is a corresponding t-test would reject the null hypothesis of a zero effect at the 5\% level.} effect of natural disasters on the performance in mathematics. The effect size is between just above zero and -0.01 standard deviations. For all subsequent periods the effect is not significant. There are some point estimates well below zero, but the uncertainty around those is relatively large. For performance in RLA, there are no significant effects.

Note that the number of observed units decreases with the distance in time from treatment. The reason for this is that in order to experience eight treated years, the county has to experience its first disaster very early. Similarly, it has to receive treatment very late to experience more than five years before treatment. As a result, the uncertainty increases with the distance in time from treatment.

For the subgroups we find some surprising results. Black students seem to perform better in RLA in the medium term after a disaster. That is, there are significantly positive results one to seven years after treatment. The effect sizes are substantial: Seven years after treatment the increase in RLA performance goes up to 0.1 standard deviations. In other words, the average black student sees an increase in performance of up to 0.1 standard deviations of the national reference cohort. For the other subgroups there are no significant effects.

Positive effects of disasters on performance are not unheard of in the literature \citep[see e.g.][]{Sacerdote_2012}. Many students have to switch schools and some may even benefit from attending a higher quality school after the disaster. Black students may disproportionally attend lower quality schools and are therefore more likely to benefit from having to switch schools. 

Figures \ref{ResultsPlotStorm} and \ref{ResultsPlotSubStorm} show the same graphs based on the storm treatment. The results look very similar. In the period of the storm there is a significant decrease in the mathematics performance of up to -0.015 standard deviations. Apart from that we find no signifcant effects.

For female students there are significant decreases in both subjects in the period of the storm. For RLA we even find a significantly negative effect one year after the storm. Similarly, economically disadvantaged students perform worse in the period of treatment and in RLA also one year after treatment. For black and hispanic students we do not find any significant effects of the storm treatment.


\begin{figure}[!h]
	\centering
	\includegraphics[scale=1]{"../Code & Data/ResultsPlotStorm.png"}
	\caption{Dynamic Treatment effects for all students: NWS storm data}
	\label{ResultsPlotStorm}
\end{figure}

\begin{figure}[!h]
	\centering
	\includegraphics[scale=1]{"../Code & Data/ResultsPlotSubStorm.png"}
	\caption{Dynamic Treatment effects for various subgroups: NWS storm data}
	\label{ResultsPlotSubStorm}
\end{figure}








