
\section{Additional Results} \label{AppendixA}

\subsection{Logistic regression for assistance applications}

Below we report logistic regression results for the applicant status, that is whether a county applied for federal disaster assistance based on the Public Assistance Applicants Program Deliveries data. More specifically, the applicant variable is $1$ if the county experienced a disaster and applied for dederal assistance, that is itappears in the Public Assistance Applicants Program Deliveries database, and $0$ otherwise. This is regressed on a few covariates, including the share of democratic votes in the 2016 election. All independent variables are as of 2016.


\begin{table}[htbp]
   \centering
   \caption{\label{ResultsLogit} Determinants of Assistance Application}
   \begin{tabular}{lcc}
      \tabularnewline\midrule\midrule
      Dependent Variables:              & Applicant      & Declared\\
      Model:                            & (1)            & (2)\\
      \midrule \emph{Variables} &   &  \\
      (Intercept)                       & -3.622         & -10.83$^{***}$\\
                                        & (3.723)        & (4.000)\\
      Share of democratic voters (2016) & -0.8362$^{**}$ &   \\
                                        & (0.3469)       &   \\
      Median Income (logs)              & 0.2939         & 0.9862$^{***}$\\
                                        & (0.3331)       & (0.3586)\\
      Poverty Rate                      & 4.033$^{**}$   & 0.2160\\
                                        & (1.575)        & (1.730)\\
      Share of single mothers           & 4.068$^{***}$  & 12.18$^{***}$\\
                                        & (1.144)        & (1.241)\\
      Share of democratic voters (2008) &                & -1.496$^{***}$\\
                                        &                & (0.3624)\\
      \midrule \emph{Fit statistics} &   &  \\
      Observations                      & 2,882          & 2,882\\
      Squared Correlation               & 0.02306        & 0.05222\\
      Pseudo R$^2$                      & 0.01966        & 0.04725\\
      BIC                               & 3,724.7        & 3,122.7\\
      \midrule\midrule\multicolumn{3}{l}{\emph{IID standard-errors in parentheses}}\\
      \multicolumn{3}{l}{\emph{Signif. Codes: ***: 0.01, **: 0.05, *: 0.1}}\\
   \end{tabular}
\end{table}





\subsection{Results for storms based on FEMA data}

Below we repeat the main analysis based on the FEMA data only considering storms. That is, all disasters that are categorized as Tornadoes, Hurricanes, or Severe Storms. The results are shown in Figure \ref{ResultsPlotFEMAStorm}.

\begin{figure}[!h]
	\centering
	\includegraphics[scale=1]{"../Code & Data/ResultsPlotFEMAStorm.pdf"}
	\caption{Dynamic Treatment effects in relative time: FEMA data (storms only)}
	\label{ResultsPlotFEMAStorm}
\end{figure}

\newpage

\section{Pre-Treatment Trends} \label{PreTrends}

Here we show plots of aggregated pre-treatment trends to justify the parallel trends assumption. Mean test scores are aggregated by cohort (year of first treatment) and relative time to treatment, and never treated units act as the control group. We only display these plots for overall test scores for both datasets, but not for subgroups. However, the plots for the subgroups look very similar.

\begin{figure}[!h]
	\centering
	\includegraphics[scale=1]{"../Code & Data/ParTrendsPlotMathematicsFEMA.pdf"}
	\caption{Aggregated mean scores in mathematics based on FEMA data in relative time to treatment}
	\label{PreTrendsMath}
\end{figure}

\begin{figure}[!h]
	\centering
	\includegraphics[scale=1]{"../Code & Data/ParTrendsPlotRLAFEMA.pdf"}
	\caption{Aggregated mean scores in RLA based on FEMA data in relative time to treatment}
	\label{PreTrendsRLA}
\end{figure}

\begin{figure}[!h]
	\centering
	\includegraphics[scale=1]{"../Code & Data/ParTrendsPlotMathematicsStorm.pdf"}
	\caption{Aggregated mean scores in mathematics based on NWS storm data in relative time to treatment}
	\label{PreTrendsMathStorm}
\end{figure}

\begin{figure}[!h]
	\centering
	\includegraphics[scale=1]{"../Code & Data/ParTrendsPlotRLAStorm.pdf"}
	\caption{Aggregated mean scores in RLA based on NWS storm data in relative time to treatment}
	\label{PreTrendsRLAStorm}
\end{figure}

\newpage



