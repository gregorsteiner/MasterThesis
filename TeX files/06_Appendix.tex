
\section{Additional Results} \label{AppendixA}

\subsection{Logistic regression for assistance applications}


\begin{table}[htbp]
   \centering
   \caption{\label{ResultsLogit} Determinants of Assistance Application}
   \begin{tabular}{lcc}
      \tabularnewline\midrule\midrule
      Dependent Variables:              & Applicant      & Declared\\
      Model:                            & (1)            & (2)\\
      \midrule \emph{Variables} &   &  \\
      (Intercept)                       & -3.622         & -10.83$^{***}$\\
                                        & (3.723)        & (4.000)\\
      Share of democratic voters (2016) & -0.8362$^{**}$ &   \\
                                        & (0.3469)       &   \\
      Median Income (logs)              & 0.2939         & 0.9862$^{***}$\\
                                        & (0.3331)       & (0.3586)\\
      Poverty Rate                      & 4.033$^{**}$   & 0.2160\\
                                        & (1.575)        & (1.730)\\
      Share of single mothers           & 4.068$^{***}$  & 12.18$^{***}$\\
                                        & (1.144)        & (1.241)\\
      Share of democratic voters (2008) &                & -1.496$^{***}$\\
                                        &                & (0.3624)\\
      \midrule \emph{Fit statistics} &   &  \\
      Observations                      & 2,882          & 2,882\\
      Squared Correlation               & 0.02306        & 0.05222\\
      Pseudo R$^2$                      & 0.01966        & 0.04725\\
      BIC                               & 3,724.7        & 3,122.7\\
      \midrule\midrule\multicolumn{3}{l}{\emph{IID standard-errors in parentheses}}\\
      \multicolumn{3}{l}{\emph{Signif. Codes: ***: 0.01, **: 0.05, *: 0.1}}\\
   \end{tabular}
\end{table}




\subsection{Dynamic Treatment Effects}

Estimation results corresponding to figures \ref{ResultsPlot} and \ref{ResultsPlotSub} are reported below.  


\begin{table}[htbp]
   \centering
   \caption{\label{MainResultsMath} Results (Mathematics)}
   \begin{tabular}{lccccc}
      \tabularnewline\midrule\midrule
      Dependent Variables: & Overall        & Black           & Hispanic        & Female   & Econ. Disadv.\\
      Model:               & (1)            & (2)             & (3)             & (4)      & (5)\\
      \midrule \emph{Variables} &   &   &   &   &  \\
      Year$=$-5            & -0.0289        & -0.0584$^{***}$ & -0.0422$^{***}$ & -0.0219  & -0.0534$^{***}$\\
                           & (0.0170)       & (0.0124)        & (0.0104)        & (0.0155) & (0.0130)\\
      Year$=$-4            & -0.0039        & -0.0125$^{*}$   & -0.0119$^{**}$  & -0.0011  & -0.0122\\
                           & (0.0093)       & (0.0065)        & (0.0053)        & (0.0087) & (0.0076)\\
      Year$=$-3            & -0.0016        & -0.0125$^{**}$  & 0.0047          & -0.0019  & -0.0041\\
                           & (0.0062)       & (0.0046)        & (0.0034)        & (0.0061) & (0.0054)\\
      Year$=$-2            & 0.0002         & -0.0140$^{***}$ & -0.0019         & 0.0012   & 0.0016\\
                           & (0.0024)       & (0.0023)        & (0.0020)        & (0.0026) & (0.0021)\\
      Year$=$0             & -0.0059$^{*}$  & -0.0017         & 0.0014          & -0.0024  & -0.0054$^{*}$\\
                           & (0.0027)       & (0.0037)        & (0.0031)        & (0.0034) & (0.0027)\\
      Year$=$1             & -0.0132$^{**}$ & -0.0034         & -0.0037         & -0.0057  & -0.0078\\
                           & (0.0052)       & (0.0064)        & (0.0032)        & (0.0063) & (0.0055)\\
      Year$=$2             & -0.0081        & -0.0019         & 0.0010          & -0.0042  & -0.0012\\
                           & (0.0094)       & (0.0073)        & (0.0062)        & (0.0096) & (0.0087)\\
      Year$=$3             & -0.0106        & -0.0104         & 0.0005          & -0.0077  & -0.0065\\
                           & (0.0130)       & (0.0116)        & (0.0085)        & (0.0131) & (0.0114)\\
      Year$=$4             & -0.0060        & -0.0065         & 0.0070          & 0.0001   & 0.0068\\
                           & (0.0160)       & (0.0128)        & (0.0107)        & (0.0157) & (0.0128)\\
      Year$=$5             & -0.0094        & -0.0098         & 0.0157          & -0.0021  & -0.0048\\
                           & (0.0187)       & (0.0155)        & (0.0111)        & (0.0180) & (0.0151)\\
      Year$=$6             & -0.0029        & 0.0059          & 0.0316$^{*}$    & 0.0048   & -0.0055\\
                           & (0.0215)       & (0.0159)        & (0.0147)        & (0.0207) & (0.0166)\\
      Year$=$7             & 0.0120         & 0.0236          & 0.0373$^{*}$    & 0.0207   & 0.0188\\
                           & (0.0247)       & (0.0210)        & (0.0176)        & (0.0256) & (0.0188)\\
      Year$=$8             & 0.0122         & -0.0072         & 0.0496$^{**}$   & 0.0197   & 0.0106\\
                           & (0.0241)       & (0.0237)        & (0.0175)        & (0.0271) & (0.0203)\\
      \midrule \emph{Fixed-effects} &   &   &   &   &  \\
      Year                 & Yes            & Yes             & Yes             & Yes      & Yes\\
      County               & Yes            & Yes             & Yes             & Yes      & Yes\\
      Grade                & Yes            & Yes             & Yes             & Yes      & Yes\\
      \midrule \emph{Fit statistics} &   &   &   &   &  \\
      Observations         & 158,110        & 65,729          & 70,149          & 150,052  & 147,520\\
      R$^2$                & 0.72307        & 0.56734         & 0.55607         & 0.67896  & 0.60003\\
      Within R$^2$         & 0.00449        & 0.01030         & 0.00554         & 0.00424  & 0.00746\\
      \midrule\midrule\multicolumn{6}{l}{\emph{Clustered (Cohort) standard-errors in parentheses}}\\
      \multicolumn{6}{l}{\emph{Signif. Codes: ***: 0.01, **: 0.05, *: 0.1}}\\
   \end{tabular}
\end{table}





\begin{table}[htbp]
   \centering
   \caption{\label{MainResultsRLA} Results (RLA)}
   \begin{tabular}{lccccc}
      \tabularnewline\midrule\midrule
      Dependent Variables: & Overall       & Black          & Hispanic     & Female                 & Econ. Disadv.\\
      Model:               & (1)           & (2)            & (3)          & (4)                    & (5)\\
      \midrule \emph{Variables} &   &   &   &   &  \\
      Year$=$-5            & -0.0031       & -0.0265$^{**}$ & -0.0202      & -0.0147                & -0.0247$^{***}$\\
                           & (0.0063)      & (0.0102)       & (0.0131)     & (0.0176)               & (0.0070)\\
      Year$=$-4            & 0.0067        & 0.0022         & 0.0001       & $-1.34\times 10^{-5}$ & 0.0047\\
                           & (0.0038)      & (0.0047)       & (0.0074)     & (0.0106)               & (0.0050)\\
      Year$=$-3            & 0.0068$^{**}$ & 0.0037         & 0.0108$^{*}$ & 0.0007                 & 0.0083$^{*}$\\
                           & (0.0030)      & (0.0028)       & (0.0053)     & (0.0077)               & (0.0041)\\
      Year$=$-2            & 0.0003        & -0.0012        & -0.0051      & -0.0007                & 0.0007\\
                           & (0.0016)      & (0.0021)       & (0.0029)     & (0.0032)               & (0.0025)\\
      Year$=$0             & 0.0006        & 0.0085$^{**}$  & 0.0016       & 0.0012                 & -0.0011\\
                           & (0.0022)      & (0.0037)       & (0.0031)     & (0.0026)               & (0.0032)\\
      Year$=$1             & 0.0003        & 0.0173$^{**}$  & 0.0108$^{*}$ & 0.0101                 & 0.0048\\
                           & (0.0039)      & (0.0067)       & (0.0058)     & (0.0064)               & (0.0056)\\
      Year$=$2             & 0.0031        & 0.0170$^{**}$  & 0.0059       & 0.0137                 & 0.0074\\
                           & (0.0048)      & (0.0058)       & (0.0076)     & (0.0097)               & (0.0064)\\
      Year$=$3             & 0.0043        & 0.0162         & 0.0085       & 0.0162                 & 0.0069\\
                           & (0.0053)      & (0.0102)       & (0.0096)     & (0.0125)               & (0.0070)\\
      Year$=$4             & 0.0038        & 0.0269$^{**}$  & 0.0142       & 0.0235                 & 0.0164$^{*}$\\
                           & (0.0067)      & (0.0119)       & (0.0131)     & (0.0152)               & (0.0074)\\
      Year$=$5             & 0.0131        & 0.0369$^{**}$  & 0.0215       & 0.0252                 & 0.0188$^{*}$\\
                           & (0.0079)      & (0.0142)       & (0.0152)     & (0.0206)               & (0.0099)\\
      Year$=$6             & 0.0135        & 0.0473$^{***}$ & 0.0290       & 0.0302                 & 0.0144\\
                           & (0.0099)      & (0.0137)       & (0.0192)     & (0.0233)               & (0.0108)\\
      Year$=$7             & 0.0237        & 0.0609$^{**}$  & 0.0403       & 0.0447                 & 0.0230\\
                           & (0.0135)      & (0.0207)       & (0.0232)     & (0.0268)               & (0.0139)\\
      Year$=$8             & 0.0234        & 0.0275         & 0.0162       & 0.0366                 & 0.0159\\
                           & (0.0167)      & (0.0240)       & (0.0286)     & (0.0293)               & (0.0201)\\
      \midrule \emph{Fixed-effects} &   &   &   &   &  \\
      Year                 & Yes           & Yes            & Yes          & Yes                    & Yes\\
      County               & Yes           & Yes            & Yes          & Yes                    & Yes\\
      Grade                & Yes           & Yes            & Yes          & Yes                    & Yes\\
      \midrule \emph{Fit statistics} &   &   &   &   &  \\
      Observations         & 169,246       & 71,155         & 74,154       & 160,296                & 157,873\\
      R$^2$                & 0.74510       & 0.59150        & 0.59889      & 0.71173                & 0.61003\\
      Within R$^2$         & 0.00353       & 0.00675        & 0.00612      & 0.00422                & 0.00395\\
      \midrule\midrule\multicolumn{6}{l}{\emph{Clustered (Cohort) standard-errors in parentheses}}\\
      \multicolumn{6}{l}{\emph{Signif. Codes: ***: 0.01, **: 0.05, *: 0.1}}\\
   \end{tabular}
\end{table}




\newpage

\section{Pre-Treatment Trends} \label{PreTrends}

Here we display plots of aggregated pre-treatment trends to justify the parallel trends assumption. Mean test scores are aggregated by cohort (year of first treatment) and relative time to treatment, and never treated units act as the control group. Figure \ref{PreTrendsMath} and \ref{PreTrendsRLA} show the results for mathematics and RLA respectively. We only display these plots for overall test scores and not for subgroups. However, the plots for the subgroups look very similar.


\begin{figure}[!h]
	\centering
	\includegraphics[scale=1]{"../Code & Data/ParTrendsPlotMathematics.png"}
	\caption{Pre trends for aggregated mean scores in mathematics}
	\label{PreTrendsMath}
\end{figure}

\begin{figure}[!h]
	\centering
	\includegraphics[scale=1]{"../Code & Data/ParTrendsPlotRLA.png"}
	\caption{Pre trends for aggregated mean scores in RLA}
	\label{PreTrendsRLA}
\end{figure}



