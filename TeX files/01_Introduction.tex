
\section{Introduction}

Natural disasters are responsible for massive economic damage and due to climate change the frequency of such disasters will increase in most regions \citep{IPCC_2021}. Therefore, it is essential to have a good understanding of the consequences. While much research has been done on the economic consequences of natural disasters, few studies have focused on the impact on education.

A causal effect of natural disasters may be driven by school closures \citep{Grewening_2020}, destroyed infrastructure, and emotional stress \citep{Vogel_2016}. To identify dynamic causal effects of experiencing natural disasters, this paper uses an event-study design. In particular, we estimate dynamic treatment effects for up to eight years after initial treatment. As a result, not only short-term but also medium to long-term effects can be found. Since treatment effects are likely very heterogenous, we use the estimator by \cite{Sun_2021}. 

This article uses standardized test data on a US county level for grades 3 through 8 in mathematics and reading language arts (RLA) to measure academic performance, covering the school years 2008/2009 to 2018/2019. This measure is very attractive as the test scores are standardized relative to a national reference cohort. Therefore, the outcomes are nationally comparable. For the same period, we obtain data on natural disasters from declarations by the Federal Emergency Management Agency and data on storms from the National Weather Service.



This article contributes to a rich literature on economic effects of natural disasters. More specifically, this paper contributes to the literature on the impact of natural disasters on education.

Some authors have investigated the link between natural disasters and human capital formation from a growth theory perspective, e.g. \cite{Cuaresma_2010} finds a strong negative relationship between geologic natural disaster risk and secondary school enrolment rates. 

Many authors working with standardized test data have focused on one particular type of disaster. There is extensive evidence for a negative effect of heat exposure on learning \cite[e.g.][]{Goodman_2020, Park_2020}. \cite{Spencer_2016} find a negative effect of hurricanes on performance in the sciences.

There is comprehensive evidence that natural disasters have negative economic effects on impacted areas \citep[e.g.][]{Deryugina_2017, Deryugina_2018} and economic difficulties may negatively influence childrens' academic performance. For instance, \cite{Rege_2011} find that paternal job loss leads to lower GPAs among affected children.

The rest of this paper is organized as follows: Section \ref{Data} introduces the data used and presents some descriptive statistics. Section \ref{EmpStrat} explains the empirical strategy. Section \ref{Results} discusses the results and section \ref{Conclusion} concludes.

