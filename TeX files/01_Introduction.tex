
\section{Introduction}

Natural disasters are responsible for massive economic damage and due to climate change the frequency of such disasters will increase in most regions \citep{IPCC_2021}. Therefore, it is essential to have a good understanding of the consequences. Few studies have focused on the impact on the education system, yet this is a crucial part of understanding the economic consequences. Negative effects in education are likely to have long-term impact on earnings potential. Inequality in disaster risk exposure could therefore exacerbate economic inequality.

A causal effect of natural disasters may be driven by school closures \citep{Grewening_2020} or lowered attendance \citep{Spencer_2016}, destroyed infrastructure, and emotional stress \citep{Vogel_2016}. In the very short term, emotional stress caused by extended housing instability, food insecurity, parental job loss, or social disconnection may be the most important factor, especially so in socially vulnerabe communities \citep{Gao_2022}. Furthermore, some forms of disasters, e.g. extreme heat, may directly impair cognitive performance \citep{Ramsey_1995}. Lastly, many students have to relocate and switch schools following a disaster. Such migration responses may also have a significant effect on the scholastic achievement of affected students \citep{Pane_2008, Sacerdote_2012}.

To identify dynamic causal effects of experiencing natural disasters, this paper uses an event-study design. In particular, we estimate dynamic treatment effects for up to eight years after initial treatment. As a result, not only short-term but also medium to long-term effects can be found. Since treatment effects are likely very heterogenous, we use the estimator by \cite{Sun_2021}. 

This article uses standardized test data on a US county level for grades 3 through 8 in mathematics and reading language arts (RLA) to measure academic performance, covering the school years 2008/2009 to 2017/2018. This measure is very attractive as the test scores are standardized relative to a national reference cohort. Therefore, the outcomes are nationally comparable. For the same period, we obtain data on natural disasters from declarations by the Federal Emergency Management Agency and data on storms from the National Weather Service.

We find strong evidence of a negative effect of disasters on mathematics performance in the same school year. Evidence of a significant effect on RLA, as well as for medium and long term effects is rather weak.
\\

\textbf{Previous Work}: This article contributes to the literature on the impact of natural disasters on the education system in the United States. Previous work has produced mixed results. Some authors find significant effects of natural disasters on the education system, while others find no or only small effects.

\cite{Holmes_2002} was among the first to study the effect of extreme weather events on academic achievement. Using a difference-in-differences approach, he finds a significantly negative effect of storms on the performance of North Carolina students. \cite{Baggerly_2008} find a statistically significant, but negligibly small effect of the 2004 hurricanes on the performance of students' test scores in Florida. \cite{Lamb_2013} study the effects of hurricane Katrina and find a significant impact on mathematics achievement in Mississippi with the greatest effects in nonpoor and nonrural schools.

\cite{Goodman_2020} find that cumulative heat exposure negatively impacts PSAT scores. At the same time, they find air conditioning to be very successful at mitigating the negative effect of heat exposure. Similarly, \cite{Park_2022} uses student level data from New York City to assess the impact of heat in high stakes exams. He finds a substantial negative effect on performance.

Many authors have focused on the role of student mobility as a consequence of natural disasters. \cite{Pane_2008} focus on students who switch schools following hurricanes Katrina and Rita and find small negative effects of displacement on test scores. Similarly, \cite{Sacerdote_2012} finds sharp declines in test scores one year after the hurricanes, but a substantial improvement three to four years after. This is largely driven by the students' tendency to switch to higher quality schools.

Most authors have focused on a single type or even a single instance of disasters. In particular, storms and heat have been investigated predominantly. The main contribution of this article is the very comprehensive dataset which covers multiple years and many different types of disasters. To the best of our knowledge, this is the first analysis using such a broad set of natural disasters to analyze the effect on academic performance.
\\

The rest of this paper is organized as follows: Section \ref{Data} gives some institutional background, introduces the data used and presents some descriptive statistics. Section \ref{EmpStrat} explains the empirical strategy. Section \ref{Results} discusses the results and Section \ref{Conclusion} concludes.

