
\section{Introduction}


A causal effect of natural disasters may be driven by school closures \citep{Grewening_2020}, destroyed infrastructure, and emotional stress \citep{Vogel_2016}. 

This article uses standardized test data on a US county level for grades 3 through 8 in mathematics and reading language arts (RLA) to measure academic performance, covering the school years 2008/2009 to 2018/2019. This measure is very attractive as the test scores are standardized relative to a national reference cohort. Therefore, the outcomes are nationally comparable.

This article contributes to a rich literature on economic effects of natural disasters. More specifically, this paper contributes to the literature on the impact of natural disasters on education.

Some authors have investigated the link between natural disasters and human capital formation from a growth theory perspective, e.g. \cite{Cuaresma_2010} finds a strong negative effect of geologic natural disaster risk on secondary school enrolment rates. 

Many authors working with standardized test data have focused on one particular type of disaster. There is extensive evidence for a negative effect of heat exposure on learning \cite[e.g.][]{Goodman_2020, Park_2020}. \cite{Spencer_2016} find a negative effect of hurricanes on performance in the sciences.

The rest of this paper is organized as follows: Section \ref{Data} introduces the data used and presents some descriptive statistics. Section \ref{EmpStrat} explains the empirical strategy. Section \ref{Results} discusses the results and section \ref{Conclusion} concludes.

