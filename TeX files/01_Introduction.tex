
\section{Introduction}

Natural disasters are responsible for massive economic damage and due to climate change the frequency of such disasters will increase in most regions \citep{IPCC_2021}. Therefore, it is essential to have a good understanding of the consequences. While much research has been done on the macroeconomic consequences of natural disasters, few studies have focused on the impact on education. This study investigates the effect of natural disasters on academic performance.

A causal effect of natural disasters may be driven by school closures \citep{Grewening_2020}, destroyed infrastructure, and emotional stress \citep{Vogel_2016}. Furthermore, some forms of natural disasters, e.g. extreme heat, may directly impair cognitive performance \citep{Ramsey_1995}.

To identify dynamic causal effects of experiencing natural disasters, this paper uses an event-study design. In particular, we estimate dynamic treatment effects for up to eight years after initial treatment. As a result, not only short-term but also medium to long-term effects can be found. Since treatment effects are likely very heterogenous, we use the estimator by \cite{Sun_2021}. 

This article uses standardized test data on a US county level for grades 3 through 8 in mathematics and reading language arts (RLA) to measure academic performance, covering the school years 2008/2009 to 2018/2019. This measure is very attractive as the test scores are standardized relative to a national reference cohort. Therefore, the outcomes are nationally comparable. For the same period, we obtain data on natural disasters from declarations by the Federal Emergency Management Agency and data on storms from the National Weather Service.

This article contributes to a rich literature on economic effects of natural disasters. More specifically, this paper contributes to the literature on the impact of natural disasters on education. Previous work has produced mixed results. Some authors find significant effects of natural disasters on the education system \citep{Holmes_2002, Cuaresma_2010, Sacerdote_2012, Goodman_2020}, while others find no or only small effects \citep{Baggerly_2008, Pane_2008}. Morover, some authors find large differences by subject \citep{Spencer_2016}.

Previous studies have overwhelmingly concentrated on a single type of disasters, or even a single instance. For example, hurricane Katrina has been the subject of many studies \citep[e.g.][]{Sacerdote_2012, Deryugina_2018}. Contrarily, we exploit a very comprehensive dataset of natural disasters, including various different types of disasters. 

The rest of this paper is organized as follows: Section \ref{Data} introduces the data used and presents some descriptive statistics. Section \ref{EmpStrat} explains the empirical strategy. Section \ref{Results} discusses the results and section \ref{Conclusion} concludes.

